% !TEX root = swt2.tex
\chapter{Software Development Process}

\section{Code and Fix}
Drauf los programmieren, sehen obs funktioniert. Wenn nicht wird es geändert,
so dass es funktioniert. \\

\textbf{Probleme:}
\begin{compactitem}
    \item Schlecht strukturierter Code (keine Planung im Voraus)
    \item Nicht systematische Verbesserungen
    \item Schwer im Team umzusetzen, da keine Einteilung existiert
    \item Kein Entwurf oder Dokumentation
    \item Wartbarkeit sehr schwierig (schlechte Doku, keine Struktur)
\end{compactitem}

\section{Prozessmodell}
Ein Prozessmodell (''Vorgehensmodell'') ist eine abstrakte Repräsentation
des Software Entwicklungsprozesses. Beschreibt den Prozess aus einer bestimmten
Perspektive.\\

\textbf{Richtlinien:}
\begin{compactitem}
    \item Welche Aktivitäten sollen ausgeführt werden (Wann? In welcher
    Reihenfolge?)
    \item Wer macht was? Rollen und Verantwortlichkeiten
    \item Welche Produkte (Artefakte, Code, Dokumente) sollen bis wann erstellt
    werden
    \item Manchmal welche Techniken und Werkzeuge verwendet werden.
\end{compactitem}
\subsection{Vorteil von Prozessmodellen}
\begin{compactitem}
    \item Klardefinierte Arbeitspakete/Aufgaben $\Rightarrow$ Bessere Verteilung
    auf Team
    \item Fortschritt feststellbar ('' Wo stehe ich in meinem Projekt '')
    \item Verbesserung der Wartbarkeit (Dokumentation)
    \item Geringeres Projektrisiko
    \item Höhere Produktqualität
    \item Explizite Phase der Qualitätssicherung
\end{compactitem}
Vorteile hängen jedoch vom konkreten Projekt und den gewählten Prozessen ab.


\section{Shitty Models}
\begin{compactitem}
    \item Wasserfallmodell
    \item V-Modell
    \item Spiralmodell (Kombination Wasserfall + V Modell)
\end{compactitem}
Können als Blöcke für Software Entwicklungs Prozesse gesehen werden.

\section{Unified Process, UP}
\subsection{Phases}
\begin{compactitem}
    \item Inception (Anfangsphase): Projektumfang klarmachen, Vision und Business
    Case
    \item Elaboration (Ausarbeitungsphase): Hauptarchitektur bauen, Elemente mit
    hohem Risiko lösen, meisten Anforderungen festlegen, allgemeinen Zeitplan
    schätzen
    \item Construktion (Konstruktionsphase): inkrementelle Implementierung der
    restlichen Elemente mit geringerem Risiko
    \item Transition (Übergabephase): Beta Tests, deployment
\end{compactitem}
\subsection{Disciplines}
\begin{compactitem}
    \item Business modelling: Verstehen und Kommunizieren der Struktur der Organisation
    wo das System verwendet wird.
    \item Requirements
    \item Design
    \item Implementation
    \item Test
    \item Deployment
\end{compactitem}
\section{Rational Unified Process RUP}
Bestimmte Implementierung des UP. Zustätzlch zu Phasen und Disciples gibt es:
\begin{compactitem}
    \item Rollen, wer macht was?
    \item Activities/Tasks, wie wird es gemacht?
    \item Artifacts/work products, was wird gemacht?
\end{compactitem}

\subsection{Best Practices}
\begin{compactitem}
    \item software iterativ entwickeln
    \item Anforderungen managen
    \item component-basierte Architekturen verwenden
    \item Software visuell modellieren
    \item Software auf qualität testen
    \item Kontrolle wird zu Software
\end{compactitem}
