\chapter{Requirements Engineering}


\section{Requirements Elicitation Technique}
\begin{compactitem}
    \item Questioning techniques
    \item Creativity techniques (Brainstorming, Analogy, Perspektivenwechsel)
    \item Retrospective techniques (Reuse, System archeology, competing systems)
    \item Observation techniques
    \item Supporting actions and techniques (Mind Maps, Workshops, User Stories,
    Use Case modelling etc...)
\end{compactitem}

\section{Concerned Based Classification / Bedürfnisbasiert}
Schwer funktionale und nicht-funktionale Anforderungen zu definieren, da sehr
schwammig. Daher Anforderungen lieber basierend auf zugrunde liegendm Bedürfniss.
\begin{compactitem}
    \item Representation
    \begin{compactitem}
        \item Operational
        \item Quantitative
        \item Qualitative: Nicht direkt verifizierbar
        \item Deklarative
    \end{compactitem}
    \item Kind
    \begin{compactitem}
        \item Functional
        \item Quality
        \item Constraint
    \end{compactitem}
\end{compactitem}


\section{Basic Writing Recommendation}
\begin{compactitem}
    \item Kurze Sätze, Ein Requirement pro Satz
    \item Aktive Sprache, wer macht was
    \item ,,Schwache'' Wörter vermeiden
    \item Glossar an Worten/Termen bereit haben
\end{compactitem}

\section{Requirement Validation Techniques}
\begin{compactitem}
    \item Inspection, Reviews, Walkthroughs: finde Fehler manuell
    \item Simulation: Aspekte des Systems simulieren.
    \item Prototyping: orientiert am Design. Stakeholder testet Szenarien im
    Prototyp
    \item Creation of system test cases
    \item Model Checking: formale Verifikatio des verwendeten Models
\end{compactitem}
