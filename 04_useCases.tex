\chapter{Use Cases}
Use Cases sind Notationen um Textuell eine Interaktion von
zwei Parteien festzuhalten. (Nutzer + System)

\section{Elementary Business Process, EBP}
Im business process engineering ist ein EBP:
\begin{itemize}
    \item Eine Aufgabe
    \item ausgeführt von einer Person
    \item an einem Ort zu einer Zeit
    \item als Antwort auf ein Business Event
    \item welches zu messbaren business Wert beisteuert
    \item und Daten in einem konsistenen Zustand hinterlässt
\end{itemize}

\section{Use case Goal Levels}
\begin{itemize}
    \item High Level Summary (oftmals ein Business process)
    \item User goal (= EBP)
    \item Sub(function)
    \item Too low (=feature oder Systemoperation)
\end{itemize}

\section{Schlüsselbegriffe}
\begin{itemize}
    \item \textbf{stakeholder}: Person/Organisation das (in)direkten Einfluss
    auf die Anforderungen des Systems hat
    \item \textbf{actor}: Entitiät mit Verhalten
    \item \textbf{primary Actor}: initiiert Interaktion
    \item \textbf{use case model}: set aller use cases und verwandter Diagramme
    \item \textbf{scenario}: eine Use Case Instanz
    \item \textbf{use case}: Sammlung ähnlicher erfolgreicher und fehlgeschlagener
    Szenarien welche einen actor beschreiben, der das System nutzt um ein Ziel
    zu erreichen.
    \item \textbf{include}: Pflicht, Base Case ist abhängig von included case.
    Beschreiben Untersequenz die immer oder manchmal passieren.
    \item \textbf{extend}: Optional, Erweiterung
\end{itemize}

\section{Use Cases finden}
\begin{enumerate}
    \item System Boundarys finden
    \item Primary Actor identifizieren
    \item Für jeden anderen actor user goals rausfinden
    \item Use cases definieren, die user goals erfüllen
\end{enumerate}

\section{Use Case Sections}
\begin{enumerate}
    \item \textbf{Preface Elements}:
    \item \textbf{Stakeholders und Interssen Liste}: Grenzt System ein und
    schlägt vor, was es tun soll
    \item \textbf{Preconditions}: Zustand immer true, wird angenommen
    \item \textbf{Postcondition}: Was true ist, wenn use case erfolgreich
    abgeschlossen
    \item \textbf{Main Success Scenario}:basic flow, beschreibt den typischen
    erfolgreichen Pfad der die Interessen der Stakeholder erfüllt
    \item \textbf{Extensions}: alternative flows, beschreibt alel anderen scenarios
    oder Zweige, egal ob Successo der fail
    \item \textbf{Special requirements}: hält fest, dass eine nicht-funktionale
    Anforderung, Qualitätsattribut oder Constraint spezifisch zu einem Use Case
    gehört.
    \item \textbf{Technology and Data Variation}: beschreibt technische Variationen
    wie Dinge gemacht werden müssen statt wie.
\end{enumerate}
