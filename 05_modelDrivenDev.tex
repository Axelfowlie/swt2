% !TEX root = swt2.tex

\chapter{Model Driven Development}

\section{Ziele und Vorteile}
\mparagraph{Ziele}
\begin{compactitem}
    \item Bessere Platform Unabhängigkeit und Kompatibilität
    \item Schnelleres Entwickeln durch Code Generierung
    \item Bessere Software qualität durch vorherige Analysen
    \item Besserer Wiederverwendung
    \item Besserer ,,Separation of concers'' durch verschiedener Modelle
    \item Komplexität nicht verringern, sondern auf einem anderne Level zu
    managen durch Abstraktion
\end{compactitem}
\mparagraph{(erhoffte) Vorteile}
\begin{compactitem}
    \item Kostenreduzierung
    \item Kürzere time to market
    \item Mehr Variabilität
\end{compactitem}

\section{Arbeiteraufteilung bei MDSD}
\begin{compactitem}
    \item \textbf{Programmer}: Nicht mehr notwendig
    \item \textbf{Domain Expert}: Entwickelt Software. Weiss wie Problem aussieht,
    was rauskommt, was zu tun ist.
    \item \textbf{Technology Expert}: Liefert Methoden und Tooling damit Domain
    Expert arbeiten kann. Überlegung von Meta Modellen, Modellierung der Plattform
    und Transformationen
\end{compactitem}

\section{Warum Modelle?}
\begin{compactitem}
    \item Verringern Komplexität bzw. Fokus auf das Wesentliche
    \item erhöht Analysierbarkeit
    \item Besser mit Komplexität umgehen
    \item Verwendbar vom Domain Expert
    \item Erhöhte Kommunikation
    \item evtl eine immer konsistene Dokumentation
\end{compactitem}

\section{Definition Modell}
Eine formale Repräsentation einer
Wiedergabe eines Originals (natürlich oder künstlich) mit dessen Attributen und
Beziehungen (Reproduction). \\
Es werden nicht alle Attribute verwendet, nur die relevanten (Abstraktion).
Wird für einen bestimten Zweck verwendet (wer, wann und wozu benutzt werden?)
(Pragmatik)

\section{Meta Modell}
Ein Meta-Modell ist ein Modell, das die Elemnte  und Beziehungen einer
Modellierungssprache sowie Einschränkungen und Regeln wie valide Modelle erstellt
werden beschreibt. Umfasst
\begin{compactitem}
    \item \textbf{Abstrakte Syntax}: Was sind die Elemente und was sind die Beziehungen.
    \item \textbf{Konkrete Syntax}: Repräsentation. Wie schreibe ich etwas hin.
    \item \textbf{Statische Semantik}: Prüfung vor konkreter Modellinstanz möglich.
    Was muss für jede Modllinstanz gelten.
    \item \textbf{Dynamische Semantik}: Was drückt das Meta ;odell aus?
\end{compactitem}

\section{Object Constraint Language}
Ausdruck von Einschränkungen und Regeln von Meta Modellen.
Genaue Spezifikationen in Transformationen nutzen.
Eine beschreibende Sprache, welche folgende Spezifikationen erlaubt:
\begin{compactitem}
    \item Invarianten
    \item Vor und Nachbedingungen
    \item Initialwerte
    \item Ableitungen
    \item Body Definitions
    \item Guards
\end{compactitem}
