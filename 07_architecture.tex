\chapter{Software Architecture}

Eine Software Architektur ist die wichtige Struktur des Systems.
Sie ist ein Set von Design Entscheidungen, die schwer rückgängig zu
machen sind oder sehr früh gefällt wurden. Umfasst die Struktur des Systems
mit Komponenten, deren Beziehungen und deren Abbildung auf ausführende Umgebungen.

\section{Strukturen, Sichten und Standpunkte}
\begin{compactitem}
    \item \textbf{View, Sicht}: Repräsentation einer ganz konkreten Menge von
    Architektur Elementen und deren Beziehungen.
    \begin{compactitem}
        \item Software Architektur Dokumentation.
    \end{compactitem}
    \item \textbf{Struktur}: Menge von Elementen selber, wie sie im System stecken.
    Tatsächliche Struktur.
    \begin{compactitem}
        \item Software Architektur
    \end{compactitem}
    \item \textbf{View Points, Standpunkte}: Gruppierung von Sichten abhängig
    ihrer Anliegen.
\end{compactitem}

Jedes Softwaresystem hat eine Softwarearchitektur, nicht jedes System hat eine
Software Architektur Dokumentation. Selbst wenn sie existiert, könnte sie
inkonsistent zur Softwarearchitektur sein.

\section{Vorteile einer expliziten Architektur}
\begin{compactitem}
    \item Stakeholder Kommunikation: Besserer Kommunikation der Beteiligten
    \item System Analyse: Analyse ob System den nicht-funktionalen Anforderungen
    gerecht wird.
    \item Large-Scale Reuse: Architektur könnte in anderne Systemen wiederverwendet
    werden. Einfachere Einbindung existierender Komponenten
    \item Project Planung: Kosten abschäzung, Milestone organisation, Abhängigkeiten
    feststellen, Änderungen besser analysieren, Staffing.
\end{compactitem}


\section{Einflussfaktoren}
\begin{compactitem}
    \item \textbf{Anforderungen} (quality req., constraints)
    \begin{compactitem}
        \item Performance, Security, Safety, Availability, Maintainability,
        Scalability
    \end{compactitem}
    \item \textbf{Re-Use} (architectres, subsystems, components, styles, patterns)
    \item \textbf{Organisation} (Team size, team number, experience, organisation structure)
\end{compactitem}

\section{Reuse Terminology}
\begin{compactitem}
    \item \textbf{Architektur Pattern}: Prinzipielle Lösungsstrukturen, die für
    ein Element der Architektur durchgängig und mit weitgehendem Verzicht auf
    ausnahmen angewandt werden.
    \item \textbf{Architektur Style}: Lösung eines wiederkehrenden Problems, das
    Anwendungen auf dem Architekturlevel findet. Es passiert die Grenzen von
    Architekturelementen und beschäftigt sich häufig mit grundlegenden architekturellen
    Aspekten.
    \item \textbf{Referenzarchitektur}: Abstrakte Software-architektur, die Strukturen
    und Typen von Software-Elementen, sowie deren erlaubte Interkation und
    Verantwortlichkeiten für einen Anwendungsbereich definiert. Die Strukturen
    sind jeweils für alle Systeme innerhalb einer Domäne anwendbar.
\end{compactitem}

\section{Layered Architecture}
\mparagraph{Design Prinzipien}
\begin{compactitem}
    \item Trennung von Concerns/Anliegen
    \item single responsibility principle
    \item Information Hiding (only what is hidden can be changed without risk)
    \item Principle of Least Knowledge
    \item Don't repeat yourself (DRY)
    \item Minimiere Design im Voraus (so weit wie berechtigt)
\end{compactitem}

\mparagraph{Schichten}
\begin{compactitem}
    \item \textbf{UI}: Repräsentation der Daten an den Nutzer. (view)
    \item \textbf{Application}:Verteilt einkommende Anfragen, kontrolliert
    den Arbeitsfluss. Gute Regel: Ein Controller/Facade pro Use Case. (Control)
    \item \textbf{Domain}: Beinhaltet das Domain Modell und die Business Logik.
    \item \textbf{Infrastructure}: Business Infrastructure, Technical Services,
    Foundation (Model)
\end{compactitem}

\mparagraph{Vorteile}
\begin{compactitem}
    \item verringert ungewollte Komplexität
    \item verbessert Veränderbarkeit
    \item Klare Trennung von Anliegen/Concerns
    \item Unabhängige Austauschbarkeit
    \item Vereinfacht Testen
\end{compactitem}
\mparagraph{Nachteile}
\begin{compactitem}
    \item Erhöht normalerweise Anzahl der Klassen, durch Facaden oder
    Daten Transfer Objekte.
\end{compactitem}


\textbf{Grenze Architektur und Design Pattern: Sobald ein Pattern die Grenze von
 Architektur-Elementen überschreitet, kann es als Architektur-Pattern betrachtet werden.
}

\section{Facade und Observer}
\mparagraph{Facade}
Bietet einheitliche und vereinfachte Schnittstelle zu einer Menge von
Schnittstellen eines Subsystems. \\
Verringert die Kopplung zu vieler Klassen.

\mparagraph{Observer}
Verhaltensmuster und dient der Weitergabe von Änderungen an einem Objekt an
von diesme Objekt abhängige Strukturen weiter.
