% !TEX root = swt2.tex
\chapter{Software Components}

Eine Komponente ist ein Baustein, der zusammengesetzt, angepasst und
eingesetzt werden, ohne sein inneres zu kennen.
Modularer Austauschbarer Teil eines Systems. \\

Ein Java Objekt kann keine Komponente sein, da das Black-Block Prinzip
bei der Vererbung verletzt wird.

\section{Component Modell}
Ein Komponentenmodell definiert:
\begin{compactitem}
    \item Was ist eine Komponenten?
    \item Wie bietet diese Komponente ihre Dienste an?
    \item Wie sind Komponenten verbunden und zusammengesetzt?
    \item Wie kommunizieren Komponenten?
    \item Wo findet man Komponenten?
\end{compactitem}

\section{Service Oriented Architecture, SOA}
Elemente aus einem Web Service System haben eine von 3 Rollen:
\begin{compactitem}
    \item Service Provider
    \item Service Broker
    \item Service Requestor
\end{compactitem}

\section{Palladio Component Model}
Ist eine Domain Spezifische Modellierungs Sprache (DSL) zur frühzeitigen
Performancevorhersagen.

\subsection{Komponentenbeschreibung, Modelle und Entwickler}
\begin{itemize}
    \item \textbf{Komponenten/Component Model}: Code der Komponenten
    \begin{compactitem}
        \item \textbf{Komponententwickler} spezifiziert und entwickelt Komponenten
        \item ... Spezifiziert Datentypen
        \item ... Baut zusammengesetzte Komponenten
        \item ... Baut parametrisierte Komponenten
        \item ... Lagert Moddeling und Implementierungs Artefakte in Repositorys
    \end{compactitem}
    \item \textbf{Komposition/Composition Model}: Verschaltung/Vernetzung verschiedene  Komponenten
    \begin{compactitem}
        \item \textbf{Softwarearchitekt}: Spezifiziert eine Architektur (System Modell)
        \item ... spezifiziert neue Komopnenten und Schnittstellen.
        \item ... Benutzt Architekturstyles und Patterns
        \item ... trifft Design Entscheidungen
        \item ... macht Performancevorhersagen
        \item ... delegiert Implementierung
        \item ... leitet den gesamten Entwicklungsprozess
    \end{compactitem}
    \item \textbf{Resourcen Environment/Deployment Model}: Hardware
    \begin{compactitem}
        \item \textbf{System Deployer}: Modelliert Resourcenumgebung (Middleware, OS, Hardware...)
        \item ... Modelliert die Allokation von Komopnenten zu Resourcen
        \item ...Wartet das System
        \item ... Richtet Resourcenumgebung ein
        \item ... Deployed Komponenten
    \end{compactitem}
    \item \textbf{Usage Profile/Usage Model}: Wieviel Benutzer? Wieviel Daten?
    Welche Daten? etc.
    \begin{compactitem}
        \item \textbf{Domain Expert}: kennt sich mit der business domain aus.
        \item ... spezifiziert Nutzerverhalten.
    \end{compactitem}
\end{itemize}

Mit den Einflussfaktoren lässt sich die Antwortzeit, die Resourcennutzung und der
Durchsatz berechnen.

\subsection{Vorteile}
\begin{compactitem}
    \item Erweiterung von Legacy Software Systemen
    \item Performancevorhersagen
    \item Analytisch lösbar
    \item Simlation
\end{compactitem}

\subsection{Sichten in Palladio}
\begin{compactitem}
    \item \textbf{Strukturel}: Informationen über statische Eigenschaften des Systems.
    Wird vom Component Developer und Softwarearchitekt modelliert. (component view, assembly view)
    \item \textbf{Verhalten}: Ablaufdiagramme von Verhalten. Wird vom Component
    Developer (und Softwarearchitekt) modelliert. (intra und inter component, behaviour )
    \item \textbf{Deployment}: Information wie wird System eingesetzt wird.
    Modelliert vom System Entwickler (allocation)
    \item \textbf{Decision}: Wird durch alle 3 Rollen modelliert
\end{compactitem}

\subsection{Service Effect Specification, SEFF}
\begin{compactitem}
        \item Beschreibt die externen sichtbaren Aktionen eines Komponente erbrachten Diensts.
        \item Abstraktion interner Verhalten
        \item Beschreibt Zusammenhang zwischen bietender und benötigender Komponentenseite
        \item Parametrisierung von Resourcenverbrauch
\end{compactitem}
