% !TEX root = swt2.tex
\chapter{Software Design}
Design Class Diagrams müssen Definitionen der Software Klassen zeigen.

\section{Responsibility-Driven Design}

Softwareobjekte sind ähnlich zu Echtwelt Objekten. Daher ist wichtige Aufgabe Software Objekten
Verantwortungen zuzuweisen. Verantwortung ist abhängig von seinem Verhalten.
\begin{compactitem}
    \item doing responsibilities: doing something itself, controlling and coordinating activities
    in other objects
    \item knowing responsibilities: knowing about private encapsulated data, knowing about related
    objects, knowing about things it can derive or calculate
\end{compactitem}

\mparagraph{Zuweisung}
Verwende Operation Contracts der System Operationen. Nutze diese zum finden von Interaktionen zwischen
Objekten.

\section{Dynamic und Static Design Model}

Startpunkt eines Design Class Diagrams.
Interaktion zwischen einzelne Objekten modellieren. Wo weise ich Verantwortungen zu?
\mparagraph{Identifizieren}
\begin{compactenum}
    \item Klassen, die in der Softwarelösung teilnehmen
    \item Zeichen Klassendiagramm dieser lassen
    \item Identifiziere Klassenmethoden durch Analyze des Interaktionsdiagramm
\end{compactenum}

Static Design Model aus Dynamic Design Model Schritt für Schritt ableitbar.

\section{General Responsibility Assignment Software Patterns, GRASP}

Abgesehen von dem Fakt, dass man ywischen Domäne und Softwareobjekten eine geringe Repräsentationsdistanz
haben möchte (Design Konzept lässt sich nahtlos vom Domänen Konzept ableiten), will man wissen,
wie man Interaktions Diagramme systematisch erstellt.
\begin{compactenum}
    \item Informations Experte
    \item Creator: Welches Objekt ist dafür zuständig X zu erzeugen?
    \item Controller: Verantwortung für Empfangen und Handhaben von System Event Nachrichten
    \item Low coupling : Versuche eine geringe Verkoppelung im System zu haben.
    \item High Cohesion
    \item Polymorphism: Composition statt Inheritance
    \item Pure Fabrication: Künstliche Klasse erfinden, wenn Domänen objekt nicht existiert.
    \item Indirection: Adapterobjekt als Mediator zwischen Objekten falls Verantwortungszuweisung zu
    high coupling oder low cohesion führt.
    \item Protected Variations: informationen verstekcen.

\end{compactenum}
