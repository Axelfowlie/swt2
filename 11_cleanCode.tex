\chapter{Clean Code}
Clean code ist:
\begin{inparaitem}
    \item Elegant und Effizient
    \item Minimale Abhängigkeiten
    \item Macht eine Sache gut
    \item Einfach und Direkt
    \item gut lesbar
    \item Einfach zu erweitern
    \item Versteckt nicht die Absichten des Designers
\end{inparaitem}

\section{SOLID}
\begin{compactenum}
    \item \textbf{Single Responsibility Principle}: Jede Klasse soll nur eine einzige Verantwortung haben.
    \item \textbf{Open Closed Principle}: Möglichkeit der Erweiterungen/Änderung (offen) ohne Verhalten
    zu ändern (geschlossen)
    \item \textbf{Liskow Substitution}: Eine Instanz einer abgeleiteten Klasse muss sich so verhalten, sodass
    jemand Denkt ein Objekt der Basisklasse vor sich zu haben, ohne durch unerwartetes Verhalten
    überrascht zu werden, wenn es sich doch um ein Objekt des Subtyps handelt. Funktionen die Pointer
    auf Basisklassen verwenden müssen Objekte von abgeleiteten Klassen verwenden können.
    \item \textbf{Interface Segregation Principle}: Nutzer sollen nicht von Schnittstellen abhängig sein,
    die sie nicht brauchen. Aufteilung grosser Interfaces. Die Aufteilung soll gemäß der Anforderungen
    der Clients an die Interfaces gemacht werden, sodass die neuen Interfaces genau auf die
    Anforderungen der einzelnen clients passen.
    \item \textbf{Dependency Inversion Principle}:
    \begin{compactitem}
        \item High Level Moduls sollen nicht von Low Level Moduls abhängen, sondern von Abstraktionen
        \item Details sollen von Abstraktionen abhängen, nicht umgekehrt
    \end{compactitem}
\end{compactenum}

\section{Regeln}
\mparagraph{Gesetz von Demeter},,don't talk to strangers''
Objekte sollen nur mit Objekten aus der unmittelbaren Umgebung kommunizieren und keine Information
über deren inneren Aufbau besitzen.

\mparagraph{Boy Scout Rule}
Lasse Code sauberer zurück als du ihn aufgefunden hast.

\mparagraph{Principle of least surprise}
Funktionen oder Klassen sollen so implementiert sein, dass das Verhalten von anderen genauso
erwartet wird.

\mparagraph{Kommentare}
Kommentare sollen folgende Funktione besitzen
\begin{compactitem}
    \item Erklärend
    \item Warnend
    \item Informativ
\end{compactitem}

\mparagraph{Don't repeat yourself, DRY}
Dupliziere keinen Codeabschnitte. Copy Pasta verringert die Wartbarkeit, Verständlichkeit und
Weiterentwicklung.

\mparagraph{Keep it simple stupid, KISS}
Halte code einfach wie möglich, aber nicht einfacher

\mparagraph{You ain't gonna need it, YAGNI}
Implementiere nur benötigte Features

\mparagraph{Single Level of Abstraction, SLA}
\begin{compactitem}
    \item Aussagen in Methoden selbe Ebene
    \item Methoden in Klassen: Depth-First
\end{compactitem}
