% !TEX root = swt2.tex
\chapter{Real-time Development und Patterns}

\section{Echtzeit Systeme}
Meistens Systeme, die ihre Umgebung überwachen und kontrollieren.
Softwaresystem, bei dem die korrekte Funktionalität abhängig vom erzeugten Ergebnis und er
Zeit abhängig ist.

\textbf{Sensoren}: Sammelt Daten von Systemumgebung\\
\textbf{Aktorik}: Verändert dei Systemumgebung\\
\textbf{Zeit ist kritisch!}: Echtzeitsysteme MÜSSEN innerhalb einer gewissen Zeit antworten.

\subsection{Stimulus Response System}
Gegeben ein Stimulus, System muss in bestimmter Zeit antworten

\textbf{periodisch}: feste Intervalle\\
\textbf{aperiodisch:}: Unvorhersehbare Zeitpunkte z.B Interrupt

\subsection{Monitoring und Kontrollsysteme}
Wichtige Klasse eines Echtzeitsystems.

\textbf{Monitor}: System macht etwas, wenn Sensorwerte nicht richtig sind\\
\textbf{Control}: System kontrolliert kontinuierlich die Aktionen abhängig des Sensorwertes.

\subsection{Echtzeit Betriebssystem}
auf Echtzeitsysteme spezialisierte Betreibssysteme.
brauchen: \\
\begin{compactitem}
    \item Echtzeit Uhr: Gibt Information für Prozess Scheduling
    \item Interrupt Handler: kümmert sich um aperiodische Anfragen
    \item Scheduler: Wählt den nächsten auszuführenden Prozess aus
    \item Resource Manager: allokiert Speicher und Prozessorresourcen
    \item Dispatcher: Starte Prozessausführung
\end{compactitem}

\mparagraph{Prozess Management}
Echtzeitsystem muss in der Lage sein \textbf{Interrupt level priority} und \textbf{Clock level priority}
zu unterstützen.


\subsection{Scheduling}
\begin{itemize}
    \item \textbf{Firist in First out, FIFO}: dynamisch, non pre-emptive, ohne Prio.
    \item \textbf{Fixed Priorities}: dynamisch, statische Prio
    \item \textbf{Earlierst-Deadline-First-Schedule, EDF}: dynamsich, dynamische Prio.
    \item \textbf{Least-Laxity-First-Scheduling, LLF}: dynamisch, dynamische Prio.
    \begin{compactitem}
        \item now + laxity + remaining processing time = deadline
        \item laxity = deadilne - now - remaining processing time. negative laxity heisst, task schafft
        es nicht bis zur Deadline
    \end{compactitem}
    \item \textbf{Time-Slice-Scheduling}: dynamisch, pre-emptive, ohne Prio
\end{itemize}


\subsection{Echtzeit System Design Prozess}
\begin{compactenum}
    \item Identifiziere Stimuli und zugehörige Antworten
    \item Definiere Zeit-constraints
    \item Wähle Ausführungsplattform
    \item Verschiedene Stimuli und generierten Antworten in Prozesse zusammenfassen. (grobe Strukturierung des Systems)
    \item Entwerfe Algorithmus für Verarbeitung
    \item Entwerfe Scheduling System
\end{compactenum}


\section{RT Patterns}
\subsection{Dimensionen der Dependability}
\begin{compactitem}
    \item Availability: Liefern wenn angefordert
    \item Reliability: Funktionieren wie spezifiziert
    \item Safety: Keine Gefahr für Mensch und Umgebung
    \item Security: Datensicherheit, Schutz vor Fremdzugriff
\end{compactitem}

\textbf{Fail-Safe-State}: Zustand in dem Gefahr ausgeschlossen werden kann.

\subsection{Fault-Error-Failure-Chain}
\begin{compactitem}
    \item \textbf{Fault}: Defekt im System, muss nicht zu Ausfall führen.
    \item \textbf{Error}: Abweichung zwischen erwartetem und tatsächlichem Verhalten innerhalb der
    Systemgrenzen.
    \item \textbf{Failure}: Verhalten, das nicht Erwartungen entspricht außerhalb der Grenzen.
    \begin{compactitem}
        \item Systematischer Fehler: Fehler bei Design oder Build Time
        \item Random Fehler: Irgendwas ist kaputt gegangen. Transient gehen weg nach ner Zeit. Persistent
        bleiben bis was gemacht wird (z.B Kaputte Hardware)
    \end{compactitem}
\end{compactitem}

\subsection{Patterns}
\textbf{Channel}: ,,Röhre'', die sequenziell Daten von einem Eingangswert zu einem Ausgangswert transformiert.

Mehrere Channels verbessern Qualität:
\begin{compactitem}
    \item \textbf{Performance}: Erhöhter Durchsatz durch mehrere identische Channels
    \item \textbf{Reliability}: Mehrere Channels können auf unterschiedilche Weise operieren um
    Fault tolerance zu erreichen.
    \item \textbf{Safety}: Erhöhte Safety durch zusätzliche fault identifiaction und Safety measures.
\end{compactitem}

Verschiedene Patterns
\begin{itemize}
    \item \textbf{Protected Single Channel}: Safety ohne Redundanz durch Error Detection auf Channel.
    \item \textbf{Homogeneous Redundacy}: Schutz gegen Random Faults, Backup Channel nötig (bei Fehler umschalten)
    \item \textbf{Triple Modular Redundacy}: Schutz gegen Random Faults. Am Ende Vergleich mit Comparator.
    \item \textbf{Heterogeneous Redundacy}:Schutz gegen Random und Systematic Faults (nur wenn der selbe Fehler
    nicht in beiden Versionen gemacht wurde). Wie Homogenous aber unabhängige voneinander entworfene
     und implementierte Channels.
    \item \textbf{Monitor-Actuator Pattern}: Schutz gegen Random und Systematic Faults wenn Fail safe
    Zustande vorhanden. Unabhängige Überprüfung durch 2. Kanal. Bei Problem Shutdown bzw in Fail safe
    Zustand überführung
    \item \textbf{Sanity Check Pattern}: Schutz gegen Random und Systematic Faults wenn Fail safe
    Zustand vorhanden. Nur grobe Abschätzung ob Hauptkanal noch richtig ist im 2. Kanal
    \item \textbf{Watchdog Pattern}: Leichter Schutz gegen Zeitbasierte Fehler und Erkennung von Deadlocks
    wenn Fail Safe State vorhanden. Channel sendet Heartbeat zu Watchdog. Kein Heartbeat = Fehler.
    \item \textbf{Safety Executive Pattern}: Safety für komplexe Systeme durch erreichen eines Fail
    Safe Zustandes.
\end{itemize}
