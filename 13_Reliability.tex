% !TEX root = swt2.tex
\chapter{Reliability and Statistical Testing}
\section{Software Reliability}

Software Reliabilty ist die Wahrscheinlichkeit einer Fehlerfreien Operation
eines Computercodes zu einem bestimmten Zeitpunkt und Input.

\begin{compactitem}
    \item \textbf{Probability of Failure on demand (POFOD)} = 1 - Reliability =
    \item \textbf{Rate of occurence of failures (ROCOF)} = Anzahl der Fehler in der
    gegebenen Zeit
    \item \textbf{Mean Time Between Failures (MTBF)} = Zeit zwischen Ausfällen
    \item \textbf{Mean Time To Recover} = Zeit bis System wieder lauffähitting
    \item \textbf{Verfügbarkei} = MTBF / MTBF+MTTR
\end{compactitem}

\subsection{Software und Hardware Verfügbarkeit}

Softwarefehler enstehen durch Bugs im Code.
Hardfehler fehler treten durch Abnutzung oder Herstellerfehler auf.

\subsection{Software Testing Überblick}
\begin{compactitem}
    \item \textbf{Funktional}: Testen aller Nutzerfunktionen der Software
    \item \textbf{Abdeckung}: Testen aller Pfade
    \item \textbf{Random}: Testfällte werden durch Zufallsauswahl gewählt.
    \item \textbf{Partition}: Eingabemenge wird in Strata unterteilt. Testcases werden daraus zufällig
    gewählt.
    \item \textbf{Statistisch}: Nutze formale zufällige Expriment Paradigma zum zufälligen Testen
    des Usage Modells der Software
\end{compactitem}

\subsection{Erwartete Anzahl an Fehlern}
\label{sub:Erwartete Anzahl an Fehlern}

\begin{itemize}
    \item $k$ Failures unter $n$ trials = $\binom{n}{k}$
    \item Wahrscheinlichkeit für $k$ Fehler in $n$ Tests = $\binom{n}{k}\hat{R}^{n-k}(1-\hat{R})^k$
    \item $\hat{R}$ ist die Reliability, $\hat{R} = \frac{n-k}{n}$
\end{itemize}

\subsection{Verlässlichkeitmodell}
\begin{compactitem}
    \item Mit der steigenden Anzahl erfolgreicher Tests steigt Konfidenz des ermittelten R.
    \item Konfidenz $C[\hat{R}] = \text{Pr}[R \geq \hat{R}] = \beta$
    \item Obere Schranke der Failure Wahrscheinlichkeit: $\tilde{p} = 1 - \sqrt[n]{(1- \beta)}$
    \begin{compactitem}
        \item \textbf{Bedingungen:} $n > 100$ Testfälle
        \item Testfälle und Testdaten sind unabhängig
        \item Fehler werden korrekt erkannt
        \item Keine Fehler während dem testen
        \item Repräsentatives Operationsprofil verwendet.
    \end{compactitem}
\end{compactitem}

\subsection{Statistisches Testen Vorraussetzung}
Statistische Testfälle werden um 3 Hauptanforderungen gebaut:
\begin{compactitem}
    \item Statistische Tests werden nach probabilistischen Modellen erstellt
    \item Tests sind statistisch unabhängig von vorherigen Testläufen
    \item Existenz eines Orakel, dass Passes und Fails bestimmt.
\end{compactitem}
