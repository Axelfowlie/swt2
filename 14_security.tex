% !TEX root = swt2.tex

\chapter{Software Security}

Unsichere Software kostet Geld aufgrund von Ausfällen, Zeit und Kosten zum behandeln von Fehlern,
Dataleaks, zusätzliche Sicherheitsmaßnahmen wie Firewall, Antivirus, Spamfilter.

\textbf{Hauptziele:}
\begin{compactitem}
    \item \textbf{Vertraulichkeit}: Persönliche und Firmendaten
    \item \textbf{Verfügbarkeit} und \textbf{Integrität} von Daten und System
\end{compactitem}
\textbf{Andere Ziele}
\begin{compactitem}
    \item User authentication
    \item Traceability und Auditing
    \item Monitoring
    \item Anonymity
\end{compactitem}

\section{Sicherheits Anforderungen finden}
\begin{compactitem}
    \item Ziele festhalten (meistens zu schwammig)
    \item Finde potentielle \textbf{misuse cases}
    \item System Kontext in Betracht nehmen
    \item Thread Modelling mit Checkliste
\end{compactitem}

\section{Prinzipien}
\begin{compactenum}
    \item Schwächstes Glied sichern
    \item Schutz über mehrere Schichten
    \item Fails sichern, Exception Handeln
    \item Sicher by default
    \item Least Priviliege (so wenig Privilegien wie möglich und so kurz wie möglich)
    \item Keine Sicherheit durch Unklarheit/obscurity
    \item Minimiere Angriffsfläche (weniger Code ist besser, vermeide ,,dead code'')
    \item Privilegierter Kern: Isolierter Code mit Sicherheits Privilegien
    \item Input-Validierung undd Ausgabe-Encoding
    \item Keine Vermischung von Daten und Code
\end{compactenum}

\section{Testing Guideline}
Security testing $neq$ Functional testing

\begin{compactenum}
    \item Fokus auf identifizierte Risiken
    \item Code Coverage Metrik um Dead Code ausfindig zu machen
    \item Code und System Reviews
    \item Evaluiere Sicherheit in jeder Phase
    \item Verschiedene Quellen für Testfälle
    \item Standart Testumgebung, dass die Basics testet
    \item Tests durch Sicherheitsexperten als Black oder White Box
\end{compactenum}

 \section{Sicherheits Probleme}
 \subsection{Race Conditions}
\textbf{Problem}: Threads interagieren, Annahme muss true für gewisse Zeit sein, was sie nicht sein muss. \\
\textbf{Lösung}: Reduziere Angriffsfenster, Nutzen von File Handler oder Descriptor. Check und Atomare
Operationen

 \subsection{Buffer Overflow}
 Angreifer kann Schadcode durch Manipulation des Buffer ausführen. Besonders gefährlich in C da
 Overflows nicht gecheckt werden.
 \subsection{SQL Injection}
 \textbf{Lösung}: Input reinigen, nach validen Inputs schauen, nicht nach Invaliden


\section{Kryptografie}

Bietet Methoden und Protokolle, wenn mehrere Parties involviert sind und mindestens eine Partei
nicht vertrauenswürdig für mindestenseine andere Partei ist.

\subsection{Klassische Ziele}
\begin{compactitem}
    \item Confidentialy
    \item Integrity
    \item Authenticity
    \item Non-repudiability
\end{compactitem}

\subsection{Pseudo Randomness}
Kryptografische MEthoden brauchen gute Zufallszahlen, aber eingbaute Pseudo Random Number Generator liefern
keine gescheiten Zufallszahlen
\begin{enumerate}
    \item Verwende kryptografisch sichere PRNG
    \item Seed muss schwer zu erraten sein und genug Entropy liefern.

\end{enumerate}
